\chapter{Conclusion and Next Steps}

For the use case investigated of wind power generation, accounting for spatio-temporal instead of only temporal correlations brought significant gains in accuracy, while no significant increase in computational costs was observed.
In fact, spatio-temporal models presented better scalability than the univariate counterparts, statistical and machine learning alike.
Despite a limited set of nodes was used, and even though the resolution in space and time were close to the decorrelation scales, clear advantages in accuracy and computational costs were observed.
The results also confirmed the potential gain in accuracy by using deep learning models in forecasting, even in univariate cases.

Further investigations can be conducted by visualization of the ground truth and the forecasts of each model in different temporal and spatial positions.
The current experiments can also be expanded in terms of number of investigated methods, so as to include other spatio-temporal approaches.
The existing pipeline could be further expanded to include other methods and to easily adapt to other use cases.
Finally, other use cases in wind power, renewables and beyond could be used to identify the potential and the limits of gains when using spatio-temporal approaches.