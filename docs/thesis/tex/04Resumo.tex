\resumo{%
  O comportamento de sistemas espacialmente adjacentes pode ser usado em alguns casos para melhor predizer o comportamento de um sistema.
Esse é o caso, em especial, quando o comportamento do objeto de estudo é regido por dependências espaço-temporais significativas, com casos de uso que permeiam diversos campos: neurociência, epidemiologia, criminologia, transporte e energias renováveis.
O eventual aumento na acurácia de predição, no entanto, frequentemente exige modelagens mais complexas, que implicam em custos de desenvolvimento maiores.
Tencionando trazer evidência que suporte tais decisões de modelagem, neste trabalho fazemos a revisão, implementação e comparação de abordagens convencionais de predição de séries temporais e de abordagens que representam o estado da arte em predição espaço-temporal.
A comparação é feita num uso de caso de predição de geração eólica \textit{onshore} de energia na Alemanha.
Os resultados indicam que a aplicação de métodos de deep learning pode implicar em ganhos de acurácia de até $25\%$ quando séries temporais são modeladas individualmente, e de até $38\%$ quando incorporamos correlações espaço-temporais.
\\[3\baselineskip]
%
\textbf{Palavras-Chave}: Análise de Séries Temporais, Predição Espaço-Temporal, Aprendizagem de Máquina, Redes Neurais, Energias Renováveis, Energia Eólica.
}