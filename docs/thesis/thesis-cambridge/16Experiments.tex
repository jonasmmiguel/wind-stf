\chapter{Experiments Settings}\label{chap:experiments}
All forecasting approaches are trained and evaluated for 33 pre-defined districts in northern Germany on a same set of 52 weeks in 2015 (52 train-test splits).
Training time windows are set each to range from 2005-01-01 to the respective the train-test split position in 2015.
Table \ref{tab:experimentssettings} describes the hyperparameters configurations deviating from the original implementations, after adaptation to our use case (e.g. graph nodes in GWNet) and manual tuning.
Table \ref{tab:experimentssettings} also summarizes the preprocessing steps performed for every method.
Quantile transformations to normal distribution were applied to every time series \cite{sklearn2020qtransformation}.
Another step was used for Holt-Winters' Exponential Smoothing experiments, for meeting a requirement particular to this method: strictly positive values for models inputs.
All of our experiments were conducted under a computer environment with 1 Intel Core i7-8700K @3.70 GHz, one NVIDIA GeForce GTX 1080 Ti GPU card, and 64 GB RAM.

The chosen set of metrics MAE, RMSE and MdRAE was chosen for the purpose of models evaluation by taking as criteria:
\begin{itemize}
    \item (a) comparability to other use cases and future research: it is desirable to have error metrics widely used in forecasting community, so as to allow comparability;
    \item (b) coherence with expected values for $\hat{y}_t$: the variable of interest CF often assume near-zero values, rendering percentage-based metrics innappropriate;
    \item (c) same scale as $y_t$: it is desirable to have error metrics having the same scale as the CF itself;
    \item (d) improvement over baseline: it is desirable to assess how much better every approach perform in comparison to a baseline (naïve) approach;
    \item (e) coherence with expected values for $e_{t, na\ddot{i}ve}$: as CF $\in [0, 1]$, then $e_{t, na\ddot{i}ve} \in [0, 1]$, as naïve model forecats are simply the last observation in training dataset of a given train-test split.
    Errors relative to this may present outliers.
    To prevent this from biasing our estimates, we favor median-based metrics such as MdRAE over mean-based such as MASE.
\end{itemize}

\begin{table}\label{tab:experimentssettings}
\centering
\setlength{\extrarowheight}{0pt}
\addtolength{\extrarowheight}{\aboverulesep}
\addtolength{\extrarowheight}{\belowrulesep}
\setlength{\aboverulesep}{0pt}
\setlength{\belowrulesep}{0pt}
\caption{Experiments settings: preprocessing steps, hyperparameters configurations deviating from the original implementations, frameworks and implementations used.}
\resizebox{\linewidth}{!}{%
\begin{tabular}{lcccccc}
\toprule
\multicolumn{1}{c}{\multirow{2}{*}{Model}} & \multicolumn{2}{c}{Preprocessing}                                                                                                        & \multicolumn{2}{c}{\multirow{2}{*}{Hyperparameters}}                                                                                                                                          & \multirow{2}{*}{Framework}                                                                 & \multirow{2}{*}{\begin{tabular}[c]{@{}c@{}}Base\\ Implementation \end{tabular}}  \\
\cline{2-3}
\multicolumn{1}{c}{}                       & \begin{tabular}[c]{@{}c@{}}Quantile\\ Transform \end{tabular} & \begin{tabular}[c]{@{}c@{}}Offsetting\\to $\mathbb{R}_{>0}$\end{tabular} & \multicolumn{2}{c}{}                                                                                                                                                                          &                                                                                            &                                                                                  \\
\hline
\rowcolor[rgb]{0.91,0.91,0.91} ARIMA       & $\checkmark$                                                  &                                                                          & \begin{tabular}[c]{@{}>{\cellcolor[rgb]{0.91,0.91,0.91}}c@{}}p\\ q\\ d \end{tabular}                   & \begin{tabular}[c]{@{}>{\cellcolor[rgb]{0.91,0.91,0.91}}c@{}}3\\ 1\\ 1 \end{tabular} & \begin{tabular}[c]{@{}>{\cellcolor[rgb]{0.91,0.91,0.91}}c@{}}statsmodels\\\cite{seabold2010statsmodels}\end{tabular} & \cite{statsmodels2020arima}                                                                              \\
HWES                                       & $\checkmark$                                                  & $\checkmark$                                                             & \begin{tabular}[c]{@{}c@{}}trend\\ seasonal\\ seasonal periods \end{tabular}                          & \begin{tabular}[c]{@{}c@{}}additive\\ multiplicative\\ 7 \end{tabular}               & \begin{tabular}[c]{@{}c@{}}statsmodels\\\cite{seabold2010statsmodels}\end{tabular}                                   & \cite{statsmodels2020hwes}                                                                              \\
\rowcolor[rgb]{0.91,0.91,0.91} NBEATS      & $\checkmark$                                                  &                                                                          & \begin{tabular}[c]{@{}>{\cellcolor[rgb]{0.91,0.91,0.91}}c@{}}hidden layer units\\epochs\\early stopping \end{tabular} & \begin{tabular}[c]{@{}>{\cellcolor[rgb]{0.91,0.91,0.91}}c@{}}128\\ 100\\20 \end{tabular} & \begin{tabular}[c]{@{}>{\cellcolor[rgb]{0.91,0.91,0.91}}c@{}}PyTorch\\\cite{pytorch}\end{tabular}     & \cite{amitesh2020nbeats_forecast}                                                                              \\
GWNet                                      & $\checkmark$                                                  &                                                                          & \begin{tabular}[c]{@{}c@{}}nodes\\output timesteps\\epochs\\early stopping\end{tabular}                & \begin{tabular}[c]{@{}c@{}}33\\7\\100\\20\end{tabular}                               & \begin{tabular}[c]{@{}c@{}}PyTorch\\\cite{pytorch}\end{tabular}                                       & \cite{xumwen2020spatial}                                                                              \\
\bottomrule
\end{tabular}
}
\end{table}