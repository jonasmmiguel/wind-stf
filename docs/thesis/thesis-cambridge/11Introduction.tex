\chapter{Introduction}
Harvesting energy from renewable sources is key in progressing towards sustainable development \cite{johansson2002energy}.
Today, the more than 340000 wind turbines worldwide generate energy matching 4\% of global electricity consumption \cite{}.
In the EU, this proportion reaches 10\% \cite{}.
The wind power sector is currently sized at US\$ 100 bi/year in worldwide investments, and it is growing \cite{}.
Wind power installed capacity doubles every 5 years, and should triplicate until 2030 to meet UN sustainable development goals \cite{}.

Two key issues hamper the renewables expansion in energy portfolios: high intermittency and low dispatchability of the predominant harvesting approaches.
Wind power generation, for instance, depends primarily on local wind speeds, which vary in interplaying, stochastic patterns in space and time: an inevitable consequence of the turbulence flows underlying weather and climate conditions \cite{pope2001turbulent}.
At the same time, unbalances in power supply and demand cause the network frequency to deviate from the conditions the consumer nodes are designed for.
This implies losses for the society e.g. due to shorter equipments lifespans, higher maintenance costs and productivity loss.
In practice, not being able to accurately know how much wind power will be harvested at a certain point in time and region means network stakeholders have to rely on unnecessarily larger safety margins for equalizing supply and demand.
Often, this is done by dispatching less environmentally friendly power plants at a higher cost and a larger environmental footprint.
Being unable to be dispatched on demand and not presenting steady yields ultimately limits the renewables share in the energy market.
This represents a challenge in the short and in the long-term.
At stake is for example the success of large-scale initiatives such as the $\textit{Energiewende}$ in Germany, which has enabled increased renewables feasibility and whose outcomes weight in the investiment decision of stakeholders worldwide.

Two principles are used to tackle the challenge of intermittency and dispatchability in renewables.
First, by increasing system capacitancy via energy storage.
Under current technology, this is mostly realized as large gravitational energy reservoirs or as large battery banks, which entail significant initial costs and equipment lifetime environmental footprint.
Forecasting power generation represents a complimentary alternative, with potentially much smaller economical and environmental costs.
For this purpose, a plethora of techniques exists across the spectrum between sole reliance on rules and exclusive reliance on data.

At one side of this spectrum are the physical model-based forecasting.
They use physical laws such as conservation of mass, energy and entropy to describe the system's behavior.
Because the phenomena at play are often dominated by chaos, physical models present high sensitivity to boundary conditions which are hard to estimate or measure, limiting their applicability.
Scalability is another key issue, as systems often present large scales and high frequencies in behavior change.
This is the case, for instance, when using Numerical Weather Prediction (NWP) models to forecast local wind speeds across regions, systems which span hundreds of kilometers, and change significantly in sub-hourly rates \cite{}.
In this context, data-driven forecasting approaches emerge as a promising alternative, especially as ubiquitous and continuous monitoring data become widely available.

Phenomena presenting high socio-economical relevance which are governed by complex dependencies of both spatial and temporal nature span well beyond renewables, and can be found in diverse domains such as epidemiology, criminology, transportation and climate science \cite{atluri2018datamining}.
Forecasting is in many cases a valuable tool for enabling efficient planning and resources allocation \cite{armstrong2002principles}.
Accurate, scalable and feasible rule-based forecasting modeling, however, often remains elusive.
As in wind power generation, data-driven approaches arise again as interesting alternatives, and their potential can be further increased by leveraging different data sources across space and time.

In this work, we use the forecasting of wind power generation as a use case to compare data-driven approaches to forecasting system's behaviour for which spatial and temporal dependencies are both non-negligible.
The aim is to bring evidence for supporting design decisions in the development of forecasting systems.
The core design choice issue addressed concerns the typical trade-off between (A) accuracy gains versus (B1) higher loss in  models reliability and (B2) higher development costs, when using more sophisticated modeling techniques.

By modeling sophistication we refer to reducing of models epistemic error by at least one of two general ways.
First, by providing to models data for additional variables \cite{}.
Second, by improving how effectively algorithms incorporate into models relevant correlations in available data \cite{}, i.e. by improving "learning efficacy".
Both ways of incorporating more correlations into models generally demand increased complexity in models and in modeling itself.
More complex models, e.g. in terms of increased number of parameters, often entail losses in models parsimony and thus also in their reliability, besides often requiring more computational resources for model inference.
Increased modeling complexity in turn implies higher demands in human resources as well, both in terms of expertise and of development time.
Using state-of-the-art modeling approaches further strains computational and human resources as algorithms and their software implementations lie in early phases of maturity.


\section{Problem Statement}
In spatio-temporal problems, observations of a variable of interest over neighboring locations present not only temporal but also spatial dependencies.
While local time series can be predicted individually using conventional univariate statistical techniques, information contained in their spatial and spatio-temporal correlations represent a potential for improving forecasting accuracies.
More sophisticated models that allow capturing these dependencies require however supporting evidence on their potential gains that justify the tipically higher development costs they entail.

\section{Our Hypothesis}
We hypothesize that, in use cases dominated by spatio-temporal dependencies, significant forecasting performance gains can be achieved by spatio-temporal, multi-variate, Machine Learning-based approaches.

\section{Our Contribution}
First, we delineate the state-of-the-art approaches for temporal and spatio-temporal forecasting in different domains, including statistical and machine learning-based approaches.
Second, we apply selected approaches for forecasting daily regional wind power generation in Germany.
By benchmarking against more conventional temporal, univariate, statistical approaches, we investigate to what extent more sophisticated modeling approaches add value in terms of accuracy in the use case of onshore wind power generation.