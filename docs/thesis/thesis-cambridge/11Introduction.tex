\chapter{Introduction}
Harvesting energy from renewable sources is key in progressing towards carbon neutrality and sustainable development in general.
Today, the more than 340000 wind turbines generate energy matching 4\% of overall electricity consumption worldwide.
In the EU, this proportion reaches 10\%.
The sector is currently sized at US\$ 100 bi/year in worldwide investments, and it is growing.
Wind power installed capacity doubles every 5 years, and it should triplicate until 2030 to meet UN sustainability goals.

Two key challenges limit the increase in renewables share in energy portfolios: high intermittency and low dispatchability of the predominant harvesting approaches.
Wind power generation, for instance, depends primarily on local wind speeds, which vary in interplaying, stochastic patterns in space and time: an inevitable consequence of the turbulence flows \cite{pope2001turbulent}).
At the same time, unbalances in power supply and demand cause the network frequency to deviate from the conditions the consumer nodes are designed for.
This implies losses for the society e.g. due to shorter equipments lifespans, higher maintenance costs and productivity loss.

In practice, not being able to accurately know how much wind power will be harvested at a certain point in time and region means network stakeholders have to rely on unnecessarily larger safety margins for equalizing supply and demand, often by dispatching less environmentally friendly power plants which often is results in higher costs and environmental footprint.
Being unable to be dispatched on demand and not presenting steady yields ultimately limits the renewables share in the energy market.
At stake is the success of large-scale initiatives such as the $\textit{Energiewende}$ in Germany, which has enabled increased renewables feasibility and whose outcomes weight in the investiment decision of stakeholders worldwide.

Two principles are used to tackle the challenge of intermittency and dispatchability in renewables.
First, by increasing system capacitancy via energy storage.
Under current technology, this is mostly realized as large gravitational energy reservoirs or as large battery banks, which entail significant initial costs and equipment lifetime environmental footprint.
In this context, forecasting power generation arises as a second, complimentary principle, with potentially much smaller economical and environmental costs.

Forecasting renewables power generation encompasses techniques from

This is less true for conventional, mostly rule-based meteorological forecasting approaches, which rely on physical models in computational fluid dynamics to predict renewables sources availability over time.




Forecasting


Phenomena presenting high socio-economical relevance which are governed by complex dependencies of both spatial and temporal nature are found in diverse domains such as epidemiology, criminology, transportation, climate science and astrophysics \cite{atluri2018datamining}.
Indeed, the ability to describe a system's behavior is particularly valuable for estimating its states ahead of time: forecasting \cite{armstrong2002principles}.
Accurate, scalable and feasible rule-based forecasting modeling, however, remains elusive in many cases.
Especially as ubiquitous and continuous monitoring data become available, data-driven approaches emerge as a promising alternative.

Conventional data-driven approaches alone, however, have often shown to add limited value in spatio-temporal forecasting \cite{makridakis2018waysforward}.
A major reason for this limitation lies on the assumptions they rely upon being typically violated in spatio-temporal settings.
Stationarity assumption most of the statistical approaches from time series analysis, while earlier machine learning methods assume data instances are independent and identically distributed (i.i.d.) \cite{atluri2018datamining}.
Recently, deep learning-based approaches have shown to be able to overcome this essentially by (a) modelling both spatial and temporal dependencies and (b) considering proximities in terms less obvious than Euclidean distances alone \cite{li2018dcrnn, liu2019st-mgcn, wu2019graphwavenet}.

In the context of renewables, being unable to accurately determine power generation ahead of time poses a major obstacle in progressing towards carbon neutrality.
This is a consequence of intermittency of power availability in renewables, heavily conditioned on weather and climate.


The intemittency in harvesting energy


Another approach

The intermmitency in  can be handled essentially either by energy storage
This ultimately hampers the expansion of wind farms and represents a loss for the society, as part of the paid overall generated power is lost, as well as for the environment, as less environment-friendly power sources have to be relied upon \cite{delarue2015intermittency}.
At stake is the success of wind power initiatives worldwide, which currently represent about US\$ 107 bi / year in investments.
such as the \textit{Energiewende} in Germany, where  this poses a major hindrance in decreasing overall carbon footprint in a sustainable fashion.
Accuracy on wind power generation forecasting hence has significant impact on both socio-economical and environmental aspects, in both short and long terms.

\section{Problem Statement}
In spatio-temporal problems, observations of a variable of interest over neighboring locations present not only temporal but also spatial dependencies.
While local time series can be predicted individually using conventional univariate statistical techniques, information contained in their spatial and spatio-temporal correlations represent a potential for improving forecasting accuracies.
More sophisticated models that allow capturing these dependencies require however supporting evidence on their potential gains that justify the tipically higher development costs they entail.

\section{Our Hypothesis}
We hypothesize that, in use cases dominated by spatio-temporal dependencies, significant forecasting performance gains can be achieved by spatio-temporal, multi-variate, Machine Learning-based approaches.

\section{Our Contribution}
First, we delineate the state-of-the-art approaches for temporal and spatio-temporal forecasting in different domains, including statistical and machine learning-based approaches.
Second, we apply selected approaches for forecasting daily regional wind power generation in Germany.
By benchmarking against more conventional temporal, univariate, statistical approaches, we investigate to what extent more sophisticated modeling approaches add value in terms of accuracy in the use case of onshore wind power generation.