\abstract{%
  In some cases, information on neighboring systems' behavior can be used to enhance the forecasting of a system's behavior.
  This is especially the case in objects of study subject to significant spatio-temporal dependencies, found in many fields such as neuroscience, epidemiology, criminology, transportation, and renewables.
  The improvement in forecasting accuracy, however, often entails losses in modeling parsimony and increases in development costs.
  Aiming to bring evidence that supports the decision under such trade-off, we assess in this work the extent of accuracy gain when using the state-of-the-art approaches in comparison to conventional time series forecasting techniques.
  We perform the review, implementation and comparison of approaches in a use case of forecasting of wind power generation in Germany.
  Our results indicate that deep learning methods can bring accuracy gains of up to $30\%$ in terms of RMSE of Capacity Factors when using temporal correlations alone and of up to $50\%$ when modeling spatio-temporal information.
\\[3\baselineskip]
%
\textbf{Keywords}: Time Series Analysis. Spatio-Temporal Forecasting. Machine Learning, Neural Networks, Graph Neural Networks, Wind Power Generation.
}