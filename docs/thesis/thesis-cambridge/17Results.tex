\chapter{Results}

Table \ref{tab:performances} summarizes the models performances in terms of means and standard deviations for each metric, both estimated over all 33 districts and 52 train-test splits, for predictions of CF.
The choice of metrics was made in accordance to the use case requirements, as described in chapter \ref{chap:experiments}.
Table \ref{tab:trainingtimes} show the overall training times for each method, normalized by the number of districts (33) and the number of train-test splits (52).
We found that GWNet models outperformed the statistical counterpart models by more than 38\% in terms of RMSE, while also presenting the shortest training times.

\begin{table}[]
\centering
\caption{Performances of different forecasting approaches for the wind power generation use case. Metrics are calculated on basis of CF values at test timestamps.}
\label{tab:performances}
\begin{tabular}{c|rrr}
\hline
Model  & \multicolumn{1}{c}{MAE} & \multicolumn{1}{c}{RMSE} & \multicolumn{1}{c}{MdRAE} \\ \hline
ARIMA  & 0.143 (0.078)           & 0.174 (0.093)            & 0.963 (0.563)             \\
HW-ES  & 0.168 (0.083)           & 0.228 (0.107)            & 1.106 (0.521)             \\
NBEATS & 0.105 (0.050)           & 0.131 (0.066)            & 0.984 (0.389)             \\
GWNet  & \textbf{0.086 (0.045)}  & \textbf{0.107 (0.056)}   & \textbf{0.812 (0.335)}    \\ \hline
\end{tabular}
\end{table}

\begin{table}[]
\centering
\caption{Computation time at training, in $s/(districts \cdot splits)$. Univariate models ARIMA, HW-ES and NBEATS were trained without paralellization. GWNet models were trained using GPU acceleration.}
\label{tab:trainingtimes}
\begin{tabular}{cr}
\hline
Model  & \multicolumn{1}{c}{\begin{tabular}[c]{@{}c@{}}Training \\ Time\end{tabular}} \\ \hline
ARIMA  & 1.12                                                                            \\
HW-ES  & 0.63                                                                            \\
NBEATS & 40.6                                                                            \\
GWNet  & 0.49                                                                            \\ \hline
\end{tabular}
\end{table}