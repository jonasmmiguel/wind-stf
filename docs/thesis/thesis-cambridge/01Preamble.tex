%%%%%%%%%%%%%%%%%%%%%%%%%%%%%%%%%%%%%%%%%%%%%%%%%%%%%%%%%%%%%%%%%%%%%%%%%%%%%%%%
%% Some custom packages
%%
\RequirePackage{amsmath}
\RequirePackage{amssymb}
\RequirePackage{xspace}
\RequirePackage{pdfpages}
\RequirePackage{multirow} %some tables
% symbol-based footnotes
\RequirePackage[symbol]{footmisc}
\renewcommand{\thefootnote}{\fnsymbol{footnote}}

%%%%%%%%%%%%%%%%%%%%%%%%%%%%%%%%%%%%%%%%%%%%%%%%%%%%%%%%%%%%%%%%%%%%%%%%%%%%%%%%
%% Fonts (like different typewriter fonts etc.)
%%
%\RequirePackage[scaled=.87]{couriers}
%\RequirePackage[T1]{fontenc}
%\renewcommand\rmdefault{psb}




%%%%%%%%%%%%%%%%%%%%%%%%%%%%%%%%%%%%%%%%%%%%%%%%%%%%%%%%%%%%%%%%%%%%%%%%%%%%%%%%
%% Style (Changing the visual style of chapter headings and stuff.)
%%
\RequirePackage{titlesec}
% [Fixes issue #34 (see https://github.com/cambridge/thesis/issues/34). Solution from: http://tex.stackexchange.com/questions/299969/titlesec-loss-of-section-numbering-with-the-new-update-2016-03-15
\RequirePackage{etoolbox}
\makeatletter
\patchcmd{\ttlh@hang}{\parindent\z@}{\parindent\z@\leavevmode}{}{}
\patchcmd{\ttlh@hang}{\noindent}{}{}{}
\makeatother
% end of issue #34 fix]
\newcommand{\PreContentTitleFormat}{\titleformat{\chapter}[display]{\scshape\Large}
{\Large\filleft\MakeUppercase{\chaptertitlename} \Huge\thechapter}
{1ex}
{}
[\vspace{1ex}\titlerule]}
\newcommand{\ContentTitleFormat}{\titleformat{\chapter}[display]{\scshape\huge}
{\Large\filleft\MakeUppercase{\chaptertitlename} \Huge\thechapter}
{1ex}
{\titlerule\vspace{1ex}\filright}
[\vspace{1ex}\titlerule]}
\newcommand{\PostContentTitleFormat}{\PreContentTitleFormat}
\PreContentTitleFormat




%%%%%%%%%%%%%%%%%%%%%%%%%%%%%%%%%%%%%%%%%%%%%%%%%%%%%%%%%%%%%%%%%%%%%%%%%%%%%%%%
%% Bibliography (special style etc.)
%%
\RequirePackage[numbers,sort&compress]{natbib}




%%%%%%%%%%%%%%%%%%%%%%%%%%%%%%%%%%%%%%%%%%%%%%%%%%%%%%%%%%%%%%%%%%%%%%%%%%%%%%%%
%% Theorems, definitions, and examples
%%
\RequirePackage{amsthm}
\theoremstyle{definition}
\newtheorem{definition}{Definition}[chapter]
%% Support for `Examples` (provides a counter for examples, the possibility to
%% label and reference them etc.)
%%
\newtheorem{example}{Example}[chapter]

%% For bold math
\RequirePackage{bm}



%%%%%%%%%%%%%%%%%%%%%%%%%%%%%%%%%%%%%%%%%%%%%%%%%%%%%%%%%%%%%%%%%%%%%%%%%%%%%%%%
%% Captions: This makes captions of figures use a boldfaced small font. 
%%
\RequirePackage[small,bf, position=top, labelsep=endash]{caption}




%%%%%%%%%%%%%%%%%%%%%%%%%%%%%%%%%%%%%%%%%%%%%%%%%%%%%%%%%%%%%%%%%%%%%%%%%%%%%%%%
%% Subfigure (note: this must be included after the `caption` package). 
%%
\RequirePackage{subfig}






%%%%%%%%%%%%%%%%%%%%%%%%%%%%%%%%%%%%%%%%%%%%%%%%%%%%%%%%%%%%%%%%%%%%%%%%%%%%%%%%
%% Graphics (we set the central folder for all included graphics to
%% `./figures/`)
%%
\graphicspath{{./figures/}}
\RequirePackage{float}

%% For drawing circles and half-filled circles
\RequirePackage{tikz}
\newcommand*\emptycirc[1][1ex]{\tikz\draw (0,0) circle (#1);}
\newcommand*\halfcirc[1][1ex]{%
  \begin{tikzpicture}
  \draw[fill] (0,0)-- (90:#1) arc (90:270:#1) -- cycle ;
  \draw (0,0) circle (#1);
  \end{tikzpicture}}
\newcommand*\fullcirc[1][1ex]{\tikz\fill (0,0) circle (#1);}



%%%%%%%%%%%%%%%%%%%%%%%%%%%%%%%%%%%%%%%%%%%%%%%%%%%%%%%%%%%%%%%%%%%%%%%%%%%%%%%%
%% Content commands
%%




%%%%%%%%%%%%%%%%%%%%%%%%%%%%%%%%%%%%%%%%%%%%%%%%%%%%%%%%%%%%%%%%%%%%%%%%%%%%%%%%
%% Glossary entries
%%
\newglossaryentry{pi}{
    name={\ensuremath{\pi}},
    sort={pi},
    description={ratio of the circumference of a circle to the diameter}
}
